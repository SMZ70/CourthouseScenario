\documentclass[12pt]{extarticle}
\usepackage{hyperref}
\hypersetup{colorlinks=true, linkcolor=black, pdfborder={0 0 0}, urlcolor=blue}
\pagestyle{empty}
\usepackage{enumitem}
\usepackage[margin=2cm, left=1.5cm, right=1.5cm, top=2cm, bottom=2cm]{geometry}

\usepackage{tocloft}

\usepackage{fancyhdr}
\pagestyle{fancy}
\fancyhf{}  % Clear header and footer
\fancyhead[L]{\thepage}
\fancyhead[C]{}
\fancyhead[R]{سناریو دادگاه}
\fancyfoot[C]{\thepage}

\usepackage{xcolor}
\usepackage{xepersian}
\settextfont{XM Yekan}

\title{
سناریوی دادگاه
	\footnote{
نسخه ارائه‌شده آزمایشی است و ممکن است در آینده تغییراتی در آن ایجاد شود.
	}
}

\author{
طراح: آری
	\\[0.2cm]
	\url{https://t.me/busybazi}
}
\date{}

\begin{document}
\maketitle

این سناریو با ۹ و ۱۱ نفر قابل اجراست. در ۹ نفر ۶ شهروند، ۲ مافیا و یک سیاست‌مدار (مستقل) حضور دارند. در نسخه ۱۱ نفره یک شهروند (بازرس) و یک مافیا (پادو) اضافه می‌شوند. شرایط برد هر کدام از سایدهای بازی در ادامه توضیح داده شده است:

\begin{itemize}
	\item[] \textbf{\textcolor{blue}{شهروند}}
	      \begin{itemize}
		      \item
		            همه مافیاها و سیاست‌مدار از بازی خارج شوند.
		      \item
		            در کی‌آس دو شهروند با یکدیگر دست بدهند.
	      \end{itemize}
	\item[] \textbf{\textcolor{purple}{سیاست‌مدار}}
	      \begin{itemize}
		      \item
		            در بازی یک شهروند، یک مافیا و سیاست‌مدار باقی بمانند.
		      \item
		            در بازی یک شهروند و یک سیاست‌مدار باقی بمانند.
		      \item
		            سیاست‌مدار با دو شهروند به کی‌آس برسد و شهروند با سیاست‌مدار دست بدهد.
	      \end{itemize}
	\item[] \textbf{\textcolor{red}{مافیا}}
	      \begin{itemize}
		      \item
		            سیاست‌مدار از بازی خارج شده باشد و تعداد مافیاها با تعداد شهروندان برابر شود.
		      \item
		            سیاست‌مدار در بازی باشد و تمامی شهروندان از بازی خارج شده‌ باشند.
		      \item
		            یک مافیا و دو شهروند به کی‌آس برسند و یک  شهروند با مافیا دست بدهد.
	      \end{itemize}
\end{itemize}


در ادامه به معرفی نقش‌های طرف‌های بازی می‌پردازیم.

\clearpage
\renewcommand{\contentsname}{نقش‌های بازی}
\setcounter{tocdepth}{1}

\tableofcontents

\setlength{\baselineskip}{1.2\baselineskip}

\clearpage
\addtocontents{toc}{
	\protect\contentsline{section}{
		\color{red}\Large
		مافیا
	}{}{}
}

\section*{تیم مافیا}
تیم مافیا در این سناریو چهار قابلیت دارد که هر شب می‌تواند فقط یکی از آن‌ها را استفاده نماید. سه قابلیت
\textbf{شات}،
\textbf{ناتویی}
و
\textbf{مذاکره}
مربوط به تیم مافیا بوده و منحصر به نقش خاصی نمی‌باشند اما قابلیت
\textbf{رشوه}
مختص نقش اولیگارش است.

\section{اولیگارش}
اولیگارش رهبر تیم مافیاست و تا زمانی که در بازی حضور دارد تصمیمات اصلی تیم شامل شات، ناتویی، رشوه و مذاکره بر عهده اولیگارش می‌باشد.
رشوه و شات توسط گرداننده اعلام نمی‌شود ولی مذاکره و ناتویی اعلام می‌شود.

\subsection{رشوه}
در این سناریو تیم مافیا می‌تواند در صورت از دست دادن یکی از یاران خود، با بازجو یا دادستان مذاکره کنند و او را به تیم خود اضافه نمایند. اما پیش‌نیاز مذاکره‌ رشوه است. رشوه بر خلاف مذاکره، می‌تواند با حضور تمام یاران مافیا در بازی نیز انجام شود. نقش‌های بازجو و دادستان در این سناریو می‌توانند رشوه دریافت کنند. فرایند رشوه به این صورت است:
\begin{itemize}
	\item
	      اولیگارش انتخاب می‌کند که به یکی از بازیکنان رشوه بدهد.
	\item
	      اگر بازیکن انتخاب‌شده بازجو یا دادستان باشد، گرداننده به اولیگارش اطلاع می‌دهد و اولیگارش به خواب می‌رود.
	\item
	      \textcolor{red}{
		      در نسخه ۱۱ نفره
	      }
	      بازرس بیدار می‌شود و بازیکنی که به او پیشنهاد رشوه شده‌است را حدس می‌زند.
	\item
	      اگر بازیکن انتخاب‌شده توسط اولیگارش دادستان یا بازجو باشد،
	      \textcolor{red}{
		      و توسط بازرس شناسایی نشده باشد،
	      }
	      آن بازیکن توسط گرداننده بیدار می‌شود و از وی سوال می‌شود که آیا رشوه را می‌پذیرد یا خیر.
	\item
	      در صورت نپذیرفتن رشوه، تا پایان روز بعد (قبل از آغاز رای‌گیری) از بازی خارج می‌شود. بازیکنی که رشوه را نپذیرفته‌است، می‌تواند هر زمان از روز بعد از گرداننده درخواست کند که وی را از بازی خارج کند. در صورت عدم اعلام، خروج اجبارا قبل از شروع فرایند رای‌گیری صورت می‌پذیرد.
\end{itemize}

\subsection{نکات رشوه}
\begin{itemize}
	\item
	      در صورتی که در اولین اقدام به رشوه، شخص اشتباه توسط اولیگارش انتخاب شود، یک بار دیگر اولیگارش قابلیت رشوه‌دادن دارد.
	\item
	      اولیگارش از موفقیت یا عدم موفقیت رشوه دوم مطلع نمی‌شود.
	\item
	      در صورت نپذیرفتن رشوه توسط بازجو یا دادستان، اولیگارش امکان رشوه مجدد را نخواهد داشت
	\item
	      شخص رشوه‌گیر مادامی که با وی مذاکره صورت نپذیرفته‌است، همچنان شهروند محسوب می‌شود. تنها زمانی که یکی از مافیاها خروج پیدا کرده باشد، تیم مافیا می‌توانند از قابلیت مذاکره استفاده کرده و شخص رشوه‌گیر را به تیم خود اضافه نمایند. در این صورت اعضای تیم مافیا به عضو مذاکره‌شده معرفی می‌شوند.
\end{itemize}

\section{وکیل}
هر شب می‌تواند ساید یک شهروند و یک مافیا را جابه‌جا نماید. این جابه‌جایی در همه‌ی استعلام‌ها از جمله استعلام~وضعیت اول روز، استعلام ساید خروجی تفنگ سرباز در روز و استعلام ژورنالیست اختلال ایجاد خواهد کرد. مدت زمان اثر این اختلال تا قبل از شروع شب بعد می‌باشد.

\begin{itemize}
	\item
	      در صورتی که وکیل به‌جای شهروند، مستقل را انتخاب نماید، هیچ اختلالی در هیچ استعلامی به وجود نخواهد آمد.
	\item
	      وکیل هر شب باید هر دو نفر درگیر در اکت خود را تغییر دهد، هم شهروند هم مافیا. این به این معنی‌ست که در صورتی که وکیل به تنهایی از تیم مافیا در بازی باقی مانده باشد، اکت وکیل یک شب در میان خواهد بود.
\end{itemize}

\section{پادو}
\textit{\textcolor{red}{
		این نقش مختص نسخه ۱۱ نفره می‌باشد.
	}}

نفر سوم تیم مافیاست. در شب معارفه اولیگارش بایستی مشخص کند که پادو جایگزین وکیل یا اولیگارش خواهد شد. در صورت خروج نقش مشخص‌شده از بازی، پادو جایگزین آن نقش می‌شود. مثلا اگر اولیگارش مشخص کند که پادو جایگزین وکیل خواهد شد، در صورت خروج وکیل، پادو تبدیل به وکیل می‌شود. اما اگر اولیگارش اول خارج شود، مادامی که وکیل در بازی حضور دارد، تغییری در نقش پادو ایجاد نخواهد شد.

\clearpage
\addtocontents{toc}{\vspace{0.5em}\hrule\vspace{0em}}
\addtocontents{toc}{
	\protect\contentsline{section}{
		\color{blue}\Large
		شهروند
	}{}{}
}
\section*{تیم شهروند}
\section{دکتر}
هر شب می‌تواند یک بازیکن را نجات دهد. بی‌نهایت‌بار در طول بازی می‌تواند خودش را نجات دهد.

\section{سرباز}
در صورت خروج از بازی با شات در شب، با رای در روز، یا به‌واسطه‌ دست نگرفتن در کیفرخواست (صرفا در صورتی که قاضی از خروج وی ممانعت به عمل نیاورد)، یک تفنگ جنگی از گرداننده دریافت می‌کند که
\textbf{بایستی}
با آن یک بازیکن را مورد هدف قرار دهد. سرباز نمی‌تواند خواستار خروج خود از بازی شود، با بازیکنی ساید بگیرد یا بازیکنی را گردن بگیرد. در صورت دیده شدن این موارد، تفنگ به او داده نخواهد شد.
\begin{itemize}
	\item
	      در صورتی که در شب شات شود، گرداننده این موضوع را صرفا به اطلاع سرباز می‌رساند و باقی اعضای شهر از این موضوع مطلع نخواهند شد.
	\item
	      اگر در روز هر بازیکنی را بزند، آن بازیکن بدون قید و شرط از بازی خارج می‌شود.
	      بازیکنی که مورد هدف پس از وصیت سرباز، وصیت می‌کند و در نهایت طرف بازیکن تیرخورده با احتساب اختلال وکیل اعلام می‌شود.
	\item
	      اگر در شب بازیکنی را بزند:
	      \begin{itemize}
		      \item
		            اگر بازیکن زده‌شده مافیا باشد، سرباز و مافیا از بازی خارج خواهند شد. در این صورت نجات دکتر روی سرباز و مافیا موثر خواهد بود.
		      \item
		            اگر بازیکن مورد هدف سیاست‌مدار باشد و از اکت خود در آن شب استفاده کرده باشد، سیاست‌مدار از بازی خارج می‌شود. دکتر می‌تواند در این حالت سیاست‌مدار یا سرباز را نجات دهد. اگر سیاست‌مدار از اکت خود استفاده نکرده باشد سرباز در صورتی که پزشک نجاتش ندهد، به‌تنهایی از بازی خارج می‌شود.
		      \item
		            اگر بازیکن مورد هدف شهروند باشد سرباز به تنهایی از بازی خارج می‌شود و دکتر قابلیت نجات سرباز را \textbf{ندارد}.
	      \end{itemize}
\end{itemize}

\clearpage
\section{دادستان}\label{prosecutor}
قبل از رای‌گیری روز اول می‌تواند درخواست کیفرخواست دهد. اگر در روز اول تقاضا ندهد و در صورتی که بازجو و دادستان در روز اول با رای از بازی خارج نشوند، کیفرخواست به اجبار در ابتدای روز دوم برگزار خواهد شد. دادستان همراه بازجو یکی از دو نقشی‌ست که می‌تواند رشوه دریافت کند.

\subsection{فرایند کیفرخواست}\label{indictment}
\begin{itemize}
	\item
	      پس از درخواست دادستان، بازجو بیدار می‌شود و دو نفر را به گرداننده معرفی می‌کند. خودش را نمی‌تواند معرفی کند.
	\item
	      بازجو به خواب رفته و لیست دونفره به دادستان داده می‌شود. دادستان می‌تواند لیست را تایید یا رد کند.
	\item
	      در صورت رد لیست، بازجو موظف است دست کم یک نفر از لیست خود را تغییر دهد.
	\item
	      لیست دوم قابل ردکردن نیست.
	\item
	      دادستان یک نفر به لیست دونفره اضافه می‌کند. خودش را نمی‌تواند وارد کیفرخواست کند.
	\item
	      بین اعضای لیست سه‌نفره کی‌آس برگزار می‌شود و شهروند مشترک با یکی از دو بازیکن دیگر دست می‌دهد. دقت شود که زمان اضافی برای صحبت به این بازیکن برای صحبت داده نمی‌شود.
	\item
	      شخصی که دست نگرفته برای قاضی صحبت می‌کند و تقاضای تجدید نظر می‌دهد.
	\item
	      قاضی می‌تواند از یکی از دو اکت خود استفاده کرده و مانع خروج شود.
	\item
	      در صورت عدم ممانعت قاضی و خروج بازیکن، ساید او توسط گرداننده اعلام می‌شود.
\end{itemize}

\section{بازجو}
نقش مکمل دادستان است و دو نفر از سه نفر فرایند کیفرخواست را تعیین می‌کند. فرایند کیفرخواست در بخش
\ref{indictment}
توضیح داده شد.
بازجو همانند دادستان می‌تواند رشوه دریافت کند.

\section{قاضی}
دو بار در طول بازی می‌تواند در فرایند خروج بازیکنان (به جز خودش) در روز مداخله نماید. انواع مداخله قاضی به شرح زیر است:
\begin{enumerate}[label=(\roman*)]
	\item در پایان کیفرخواست از خروج بازیکنی که دست نگرفته جلوگیری کند.
	\item پس از رای‌گیری ثانویه، از خروج بازیکنی که رای خروج آورده جلوگیری کند.
	\item در صورت تساوی آرای خروج، دستور به خروج یکی از بازیکنان یا دستور به
	      باقی‌ماندن هر دو بدهد. در صورت عدم دخالت قاضی، سابقه تعیین‌کننده خواهد بود.
\end{enumerate}

\section{ژورنالیست}
استعلام دو نفر را می‌گیرد و در صورت هم‌طرف‌بودن لایک و در غیر این صورت دیسلایک
می‌گیرد.

\subsection{نکات استعلام ژورنالیست}
\begin{itemize}
	\item اختلال ایجادشده توسط وکیل در استعلام ژورنالیست موثر است.
	\item سیاست‌مدار با هر بازیکنی هم‌ساید محسوب می‌شود.
	\item ژورنالیست امکان اسپم دارد و می‌تواند هر شب بازیکن‌های تکراری را
	      انتخاب کند.
\end{itemize}

\section{بازرس}
\textit{\textcolor{red}{
		این نقش مختص نسخه ۱۱ نفره می‌باشد.
	}}

توانایی مقابله با رشوه را دارد. در صورتی که مافیا تصمیم به رشوه‌‌دادن کنند، گرداننده بازرس را بیدار کرده و از وی می‌خواهد که یک نفر را از قرار گرفتن در شرایط رشوه نجات دهد. در صورت بازرسی درست، اکت رشوه برای مافیا می‌سوزد. مافیا از موفقیت یا عدم موفقیت بازرسی مطلع نمی‌شود.

\clearpage
\addtocontents{toc}{\vspace{0.5em}\hrule\vspace{0em}}
\addtocontents{toc}{
	\protect\contentsline{section}{
		\color{purple}\Large
		مستقل
	}{}{}
}
\section*{ساید مستقل}
\section{سیاست‌مدار}
هر شب می‌تواند رای یک بازیکن از جمله خودش را گرفته و به یک بازیکن دیگر از جمله
خودش بدهد. هر شب بازیکنی که رای از او گرفته می‌شود و بازیکنی که رای به او داده می‌شود
بایستی عوض شوند.

\textbf{مثال:}
\begin{itemize}
	\item اگر در یک شب رای بازیکن ۱ را بگیرد و به ۲ بدهد. در شب بعد:
	      \begin{itemize}
		      \item \textbf{نمی‌تواند}
		            رای بازیکن ۱ را
		            \underline{بگیرد}
		            و به ۳ بدهد.
		      \item \textbf{نمی‌تواند} رای بازیکن ۴ را بگیرد و به ۲ \underline{بدهد}.
		      \item \textbf{می‌تواند} رای بازیکن ۲ را بگیرد و به ۱ بدهد.
		      \item
		            \textbf{می‌تواند}
		            رای بازیکن ۲ را بگیرد و به ۳ بدهد.
		      \item
		            \textbf{می‌تواند}
		            رای بازیکن ۳ را بگیرد و به ۱ بدهد.
		      \item
		            \textbf{می‌تواند}
		            رای بازیکن ۳ را بگیرد و ۴ بدهد.
		      \item
		            $\dots$
	      \end{itemize}
\end{itemize}

سیاست‌مدار در شب در صورتی که از اکت خود استفاده نکند
نامیراست. در این صورت، شات مافیا، ناتویی مافیا و شات سرباز در شب رویش
نمی‌نشیند.

در صورتی که همه مافیاها از بازی خارج‌شده باشند. سیاست‌مدار می‌تواند ناتویی کند
و با حدس صحیح نقش شهروندان، آنان را از بازی خارج نماید.



\clearpage
\section*{نکات گردانندگی}
\begin{itemize}
	\item ترتیب بیدارشدن نقش‌ها به صورت زیر است:
	      \begin{itemize}
		      \item وکیل
		      \item تیم مافیا
		      \item ژورنالیست
		      \item سرباز
		      \item پزشک
		      \item سیاست‌مدار
		      \item دادستان (در صورت عدم برگزاری کیفرخواست در روز)
		      \item بازجو (در صورت عدم برگزاری کیفرخواست در روز)
		      \item دادستان (در صورت عدم برگزاری کیفرخواست در روز)
	      \end{itemize}
	\item بازجو و دادستان هیچ‌کدام نمی‌توانند خودشان را وارد کیفرخواست کنند.
	\item در صورتی که تمام مافیاها خارج شده باشند، سیاست‌مدار می‌تواند ناتویی کند. این موضوع بایستی به وضوح در شب به همه بازیکنان اعلام شود. در این صورت سیاست‌مدار هم‌چنان اجازه جابه‌جاکردن رای را دارد.
	\item تعداد چالش‌های مجاز برای هر بازیکن در روز اول یک چالش و در مابقی روزها دو چالش است.
	\item بازیکن خارج‌شده در روز به هر نحوی (به جز اخراج انضباطی) بایستی زمانی دریافت کند که از قاضی تقاضای تجدید نظر کند.
	\item در صورت خروج سرباز در روز و عدم ممانعت قاضی از خروج او، ترتیب اتفاقات به صورت زیر است:
	      \begin{itemize}
		      \item سرباز وصیت می‌کند.
		      \item گرداننده به سرباز تیر جنگی می‌دهد و سرباز شلیک می‌کند.
		      \item سرباز خارج می‌شود.
		      \item بازیکن تیر خورده وصیت می‌کند.
		      \item طرف بازیکن تیرخورده با احتساب اختلال احتمالی وکیل اعلام می‌شود.
	      \end{itemize}
\end{itemize}

\end{document}
